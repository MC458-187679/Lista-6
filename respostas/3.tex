\begin{theorem}[subestrutura ótima]
    Sejam $k > 0$ e $n$ um inteiro com representação mínima $n = t_0 + t_1 \cdot 3 + \cdots + t_k \cdot 3^k$. Então, $t_1 + t_2 \cdot 3 + \cdots + t_k \cdot 3^{k-1}$ é uma representação mínima de $\dfrac{n - t_0}{3}$.
\end{theorem}

\begin{proof}[\textbf{Demonstração}]
    Considere $m = (n - t_0) / 3$ e sua representação mínima tem tamanho $l^*$. Suponha agora uma representação $m = s_0 + s_1 \cdot 3 + \cdots + s_l \cdot 3^l$ que não é ótima, isto é, $l < l^*$. Logo, temos que:
    \begin{align*}
        n &= 3 m + t_0 \\
        &= t_0 + 3 \left(s_0 + s_1 \cdot 3 + \cdots + s_l \cdot 3^l\right) \\
        &= t_0 + s_0 \cdot 3 + s_1 \cdot 3^2 + \cdots + s_l \cdot 3^{l+1} \\
        &= z_0 + z_1 \cdot 3 + z_2 \cdot 3^2 + \cdots + z_{l + 1} \cdot 3^{l+1}
    \end{align*}
\end{proof}

\begin{proof}
    Caso base: $n = 0$. Logo, $t_0 = 0$ e $k = 0$.

    Passo indutivo:
\end{proof}

\begin{proof}
    Suponha $n$ tal que para todo $m < n$ é único.

    Caso $n = 0$: Considere uma representação de tamanho $k \in \natural$ para 0. Vamos provar por indução em $k$ que todos os seus termos são 0. A unicidade da representação do 0 decorre disso.

    Caso base: $k = 0$. Logo, $0 = \sum_{i = 0}^0 3^i t_i = t_0$, ou seja, seu único termo $t_0 = 0$.

    Passo indutivo: Suponha $k$ tal que .... Logo, para $k + 1$ temos a representação $0 = \sum_{i = 0}^{k + 1} 3^i t_i$.

    Caso $t_0 = -1$: então $\sum_{i = 1}^{k + 1} 3^i t_i = \sum_{i = 0}^k 3^{k + 1} t_{i + 1} = 3 \sum_{i = 0}^k 3^i t_{i + 1} = 1$. Logo, um espaço $\sum_{i = 0}^k 3^i t_{i + 1} = \frac{1}{3} \not\in \integer$, que é impossível dado que $3^i, t_{i + 1} \in \integer$ e $k$ é finito.

    Caso $t_0 = 1$: Similar

    caso $t_0 = 0$: então, $0 = 0 + \sum_{i = 0}^k 3^i t_{i + 1}$, pela hipótese indutiva, também $t_i = 0$ para $1 \leq i \leq k + 1$.

    Caso $n > 0$: Suponha duas representações $n = \sum_{i = 0}^{k_1} 3^i t_i$ e $n = \sum_{i = 0}^{k_2} 3^i s_i$. Seja $k = \max\{k_1, k_2\}$ e considere $t_i = s_j = 0$ para $i > k_1$ e $j > k_2$ de forma que $n = \sum_{i = 0}^k 3^i t_i$ e $n = \sum_{i = 0}^k 3^i s_i$.

    \begin{align*}
        m_1 = \frac{n - t_0}{3} = \sum_{i = 0}^k 3^i t_{i + 1} && m_2 = \frac{n - s_0}{3} = \sum_{i = 0}^k 3^i s_{i + 1}
    \end{align*}

    Como $m_1, m_2 \in \natural$, então $n - t_0 \equiv n - s_0 \dmod{3}$. Logo, $t_0 \equiv s_0 \dmod{3}$, ou seja, $t_0 - s_0 = 3 x$ para um $x \in \integer$. Como $t_0, s_0 \in \{-1, 0, 1\}$, então $-3 < -2 \leq t_0 - s_0 \leq 2 < 3$, portanto, $-1 < x < 1$ ou seja, $x = 0$ e $t_0 = s_0$. Assim, $m_1 = m_2$, e pela hipótese indutiva, $t_i = s_i$ para $1 \leq i \leq k$. Como $t_0 = s_0$, o teorema segue.
\end{proof}

% ~

% Considere que $C_i$ é o menor custo de um percurso da aldeia $A_i$ para a $A_n$, sendo $1 \leq i < n$. Como os subpercursos são ótimos, podemos considerar todas as possíveis paradas $A_{i < j \leq n}$ com seus percursos ótimos até $A_n$, tomando o menor deles como parte do percurso ótimo de $A_i$. Note que se $i = n$, então não resta nenhuma aldeia no percurso, ou seja, $C_i = 0$. Assim,
% \begin{align*}
%     C_i &= \min_{i < j \leq n}\left\{t_{i, j} + C_j\right\} \\
%     C_n &= 0
% \end{align*}

% Nessa relação, o custo ótimo $C_i$ depende apenas dos custos $C_j$ das aldeias seguintes, com $i < j$, já que $t_{i, j} > 0$ e não é possível alugar uma canoa de $A_{j > i}$ para $A_i$. Assim, podemos calcular os custo a partir da última aldeia $A_n$, sem necessidade de recursão ou memorização.

% \itemdsep
% \subsection{b}

\begin{codebox}
    \Procname{$\proc{Mínimo-Dígitos}(n)$}
    \li $k \gets -1$
    \li
    \li \kw{enquanto} $n > 0$
        \Do
    \li     $k \gets k + 1$
    \li     \kw{se} $k \modulo 3 \isequal 0$ \kw{então}
            \Do
    \li         $n \gets n / 3$
            \End
    \li     \kw{senão}, \kw{se} $k \modulo 3 \isequal 1$ \kw{então}
            \Do
    \li         $n \gets (n - 1) / 3$
            \End
    \li     \kw{senão} ~ ~ ~ \Comment{$k \modulo 3 \isequal 2$}
            \Do
    \li         $n \gets (n + 1) / 3$
            \End
        \End
    \li
    \li \kw{retorna} $k$
\end{codebox}

\begin{codebox}
    \Procname{$\proc{Mínimo-Dígitos}(n)$}
    \li \kw{se} $n \isequal 1$ \kw{então}
        \Do
    \li     \kw{retorna} $0$
        \End
    \li
    \li \kw{senão}, \kw{se} $k \modulo 3 \isequal 0$ \kw{então}
        \Do
    \li     \kw{retorna} $\proc{Mínimo-Dígitos}(n / 3)$
        \End
    \li
    \li \kw{senão}, \kw{se} $k \modulo 3 \isequal 1$ \kw{então}
        \Do
    \li     \kw{retorna} $\proc{Mínimo-Dígitos}((n - 1) / 3)$
        \End
    \li
    \li \kw{senão} ~ ~ ~ \Comment{$k \modulo 3 \isequal 2$}
        \Do
    \li     \kw{retorna} $\proc{Mínimo-Dígitos}((n + 1) / 3)$
        \End
\end{codebox}

% \itemdsep
% \subsection{c}

% Vamos considerar que as linhas \ref{linha:a1:1} e \ref{linha:a1:2} executam em tempo constante $a_1$, que a \ref{linha:a2:1} e a \ref{linha:a2:2} executam em $a_2$ e que \ref{linha:a3:1} e \ref{linha:a3:2} são em tempo $a_3$. Assim, podemos descrever o tempo de execução do algoritmo por:
% \begin{align*}
%     T(n) &= a_1 + \sum_{i = 1}^{n - 1}\left(a_2 + \sum_{j = i + 1}^{n - 1} a_3\right) \\
%     &= a_1 + \sum_{i = 1}^{n - 1} a_2 + a_3 \sum_{i = 1}^{n - 1} n - a_3 \sum_{i = 1}^{n - 1} i - a_3 \sum_{i = 1}^{n - 1} 1 \\
%     &= a_1 + a_2 (n - 1) + a_3 n (n - 1) - a_3 \frac{n (n - 1)}{2} - a_3 (n - 1) \\
%     &= \frac{a_3}{2} n^2 + \frac{2 a_2 - 3 a_3}{2} n + a_1 - a_2 + a_3
% \end{align*}

% Ou seja, $T(n) \in \Theta\left(n^2\right)$, como requerido. Além disso, o único espaço adicional é do vetor $C$ de tamanho $n - 1$. Então, a complexidade de espaço será dada por:
% \begin{align*}
%     E(n) &= n - 1 + \Theta(1) = \Theta(n)
% \end{align*}

\[
    T(n) = T(n / 3) + \Theta(1) \in \Theta(\log(n))
\]
