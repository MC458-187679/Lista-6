Considere um deslocamento $T = (t_1, \ldots, t_n)$ como uma sequência, com $t_i = -1$ ou $t_i = +1$ para todo $i \leq n$, de forma que $P_T = (p_1 + t_1 k, \ldots, p_n + t_n k)$ é uma das $2^n$ possíveis sequências construídas a partir de $P$ e $k$.

\itemdsep

\begin{theorem}[subestrutura ótima]
    Seja $T = (t_1, t_2, \ldots, t_n)$ um deslocamento que gera a sequência $P_T$ com desequilíbrio mínimo. Se a subsequência $P' = (p_2, \ldots, p_n)$ contém o máximo e o mínimo de $P$, então o deslocamento $T' = (t_2, \ldots, t_n)$ também gera uma sequência $P'_{T'}$ de desequilíbrio mínimo.
\end{theorem}

\itemdsep

\begin{theorem}[escolha gulosa]
    Seja $T = (t_1, \ldots, t_n)$ um deslocamento que gera a sequência $P_T$ com desequilíbrio mínimo, $p_{\max} = \max_{1 \leq i \leq n} p_i$ e $p_{\min} = \min_{1 \leq i \leq n} p_i$. Então,
    \[
        t_1 = \begin{cases}
            +1, & \text{se } \abs{p_1 - p_{\max}} < \abs{p_1 - p_{\min}} \\
            -1, & \text{caso contrário}
        \end{cases}
    \]
\end{theorem}

\itemdsep

\begin{codebox}
    \Procname{$\proc{Desequilíbrio-Mínimo}(P, n)$}
    \li \kw{se} $n \isequal 1$ \kw{então}
        \Do
    \li     \kw{retorna} 0
        \End
    \li
    \li $\id{p-min} \gets P[1]$
    \li $\id{p-max} \gets P[1]$
    \li \kw{para} $i \gets 2$ \kw{até} $n$
        \Do
    \li     $\id{p-min} \gets \min(\id{p-min}, P[i])$
    \li     $\id{p-max} \gets \max(\id{p-max}, P[i])$
        \End
    \li
    \li $\id{p-inf} \gets \id{p-min}$
    \li $\id{p-sup} \gets \id{p-max}$
    \li \kw{para} $i \gets 1$ \kw{até} $n$
        \Do
    \li     \kw{se} $\id{p-max} - P[i] > P[i] - \id{p-min}$ \kw{então}
            \Do
    \li         $\id{p-inf} \gets \min(\id{p-inf}, P[i])$
            \End
    \li     \kw{senão}
            \Do
    \li         $\id{p-sup} \gets \min(\id{p-sup}, P[i])$
            \End
        \End
    \li
    \li $\id{max-diff} \gets \min(\id{p-sup} - \id{p-min}, \id{p-max} - \id{p-inf})$
    \li \kw{retorna} $\lceil \id{max-diff} / 2 \rceil$
\end{codebox}

\itemdsep

Podemos ver que todos os laços na função executam no máximo $n$ vezes. Então, a complexidade de tempo do algoritmo é $T(n) \in \Theta(n)$. Como não é usado nenhum armazenamento adicional, o espaço tem complexidade constante, $E(n) \in \Theta(1)$.
