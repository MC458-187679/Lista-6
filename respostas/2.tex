\def\pmin{\ensuremath{p_{\min}}\xspace}
\def\pmax{\ensuremath{p_{\max}}\xspace}
\def\pinf{\ensuremath{p_{\mathrm{inf}}}\xspace}
\def\psup{\ensuremath{p_{\mathrm{sup}}}\xspace}

\begin{theorem}[subestrutura ótima]
    Sejam $k$ e $S = (s_1, \ldots, s_n)$ uma solução de desequilíbrio mínimo. Considere também que $\pmin < \pmax$ são o mínimo e máximo de $P$. Então, para todo $1 \leq i \leq n$,
    \[
        s_i = \begin{cases}
            p_i + k & \text{se } \abs{p_i - \pmax} > \abs{p_i - \pmin} \\
            p_i - k & \text{se } \abs{p_i - \pmax} < \abs{p_i - \pmin} \\
            p_i \pm k & \text{caso contrário}
        \end{cases}
    \]
\end{theorem}

\begin{proof}
    Suponha que exista um $i$ tal que $\abs{p_i - \pmax} > \abs{p_i - \pmin}$, mas $s_i = p_i - k$. Seja $S'$ a solução tal que $s'_i = p_i + k$ e $s'_j = s_j$ para todo $j \ne i$. Se $p_i = \pmin$, então $s_i$ é o menor valor de $S$.

    Nesse caso, já que $k \leq \left\lceil \pmax - \pmin / 2 \right\rceil$ para ser ótimo, temos que o mínimo de $S'$ deverá ser maior que o de $S$, sem mudar o máximo. Então, $D(S') < D(S)$, contradizendo a suposição inicial.

    Além disso, mesmo se $p_i > \pmin$, como ele está mais próximo do mínimo, o novo máximo em $S'$ terá um diferença menor que o mínimo, em relação a $S$. Isto é, $s'_{\max} - s_{\max} \leq s'_{\min} - s_{\min}$, ou seja, $D(S') \leq D(S)$, que é impossível.

    Logo, $s_i = p_i + k$. De forma similar, temos que $s_j = p_j - k$ quando $\abs{p_j - \pmax} < \abs{p_j - \pmin}$. Se $\abs{p_l - \pmax} = \abs{p_l - \pmin}$, isto é, $p_l = (\pmax + \pmin) / 2$, as duas soluções são válidas.
\end{proof}

\itemdsep

\begin{theorem}[escolha gulosa]
    Seja $\pinf = \max\{p_i \mid \pmax - p_i > p_i - \pmin\}$ e $\psup = \min\{p_i \mid \pmax \leq p_i - \pmin\}$. Então, com
    \[
        k = \left\lceil\frac{\min\left\{\psup - \pmin, \pmax - \pinf\right\}}{2}\right\rceil
    \]

    Pode-se gerar uma sequência de desequilíbrio mínimo.
\end{theorem}

\begin{proof}
    Para valores pequenos de $k$, $s_{\mathrm{min}} = \pmin + k$ continua sendo o mínimo da sequência e $s_{\mathrm{max}} = \pmax - k$, o máximo. No entanto, quando $k > (\psup - \pmin) / 2$ e $k > (\pmax - \pinf) / 2$, o mínimo se torna $s_{\mathrm{sup}} = \psup - k$ e o máximo, $s_{\mathrm{inf}} = \pinf + k$.

    Então, na faixa de transição, em que
    \[
        \frac{\psup - \pmin}{2} \leq k \leq \frac{\pmax - \pinf}{2}
    \]

    Ou
    \[
        \frac{\pmax - \pinf}{2} \leq k \leq \frac{\psup - \pmin}{2}
    \]

    Pode-se alcançar o desequilíbrio mínimo. Desse forma, o $k$ escolhido como \\$\left\lceil (\min\left\{\psup - \pmin, \pmax - \pinf\right\})/2\right\rceil$ é semrpe válido.
\end{proof}

\itemdsep

\begin{codebox}
    \Procname{$\proc{Desequilíbrio-Mínimo}(P, n)$}
    \li $\id{p-min} \gets P[1]$
    \li $\id{p-max} \gets P[1]$
    \li \kw{para} $i \gets 2$ \kw{até} $n$
        \Do
    \li     $\id{p-min} \gets \min(\id{p-min}, P[i])$
    \li     $\id{p-max} \gets \max(\id{p-max}, P[i])$
        \End
    \li
    \li $\id{p-inf} \gets \id{p-min}$
    \li $\id{p-sup} \gets \id{p-max}$
    \li \kw{para} $i \gets 1$ \kw{até} $n$
        \Do
    \li     \kw{se} $\id{p-max} - P[i] > P[i] - \id{p-min}$ \kw{então}
            \Do
    \li         $\id{p-inf} \gets \max(\id{p-inf}, P[i])$
            \End
    \li     \kw{senão}
            \Do
    \li         $\id{p-sup} \gets \min(\id{p-sup}, P[i])$
            \End
        \End
    \li
    \li $\id{max-diff} \gets \min(\id{p-sup} - \id{p-min}, \id{p-max} - \id{p-inf})$
    \li \kw{retorna} $\lceil \id{max-diff} / 2 \rceil$
\end{codebox}

\itemdsep

Podemos ver que todos os laços na função executam no máximo $n$ vezes. Então, a complexidade de tempo do algoritmo é $T(n) \in \Theta(n)$. Como não é usado nenhum armazenamento adicional, o espaço tem complexidade constante, $E(n) \in \Theta(1)$.
