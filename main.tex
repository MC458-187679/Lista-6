\documentclass[a4paper, 14pt]{extarticle}

\usepackage[brazilian]{babel}
\usepackage[utf8]{inputenc}
\usepackage[T1]{fontenc}
\usepackage[margin=1.5cm,top=1.8cm,noheadfoot=true]{geometry}

\usepackage{float, pgf, caption, subcaption}

% \input{secoes}
% \input{teorema}
\makeatletter

\usepackage{amsthm, amsmath, amssymb, bm, mathtools}
\usepackage{enumitem, etoolbox, xpatch}
% \usepackage[mathcal]{euscript}
% \usepackage[scr]{rsfso}
\usepackage{mathptmx}
\usepackage{relsize, centernot, tikz, xcolor}


%%%% QED symbols %%%%
\def\qed@open{\ensuremath{\square}}
\def\qed@open@small{\ensuremath{\mathsmaller\qed@open}}

\def\qed@fill{\ensuremath{\blacksquare}}
\def\qed@fill@small{\ensuremath{\mathsmaller\qed@fill}}

\definecolor{qed@gray}{gray}{0.8}
\def\qed@gray{\ensuremath{\color{qed@gray}\blacksquare}}
\def\qed@gray@small{\ensuremath{\color{qed@gray}\mathsmaller\blacksquare}}

\def\qed@both@cmd#1#2{\begin{tikzpicture}[baseline=#2]
    \draw (0,0) [fill=qed@gray] rectangle (#1,#1);
\end{tikzpicture}}
\def\qed@both{\qed@both@cmd{0.6em}{0.2ex}}
\def\qed@both@small{\qed@both@cmd{1ex}{0ex}}

\def\showAllQED{
    \qed@open ~ \qed@open@small \\
    \qed@fill ~ \qed@fill@small \\
    \qed@gray ~ \qed@gray@small \\
    \qed@both ~ \qed@both@small
}

%% escoha do QED %%
\renewcommand{\qedsymbol}{\qed@fill@small}

% marcadores de prova
\newcommand{\direto}[1][~]{\ensuremath{(\rightarrow)}#1}
\newcommand{\inverso}[1][~]{\ensuremath{(\leftarrow)}#1}

% fontes
% conjunto potencia
\DeclareSymbolFont{boondox}{U}{BOONDOX-cal}{m}{n}
\DeclareMathSymbol{\pow}{\mathalpha}{boondox}{"50}

% somatorio
\DeclareSymbolFont{matext}{OMX}{cmex}{m}{n}
\DeclareMathSymbol{\sum@d}{\mathop}{matext}{"58}
\DeclareMathSymbol{\sum@t}{\mathop}{matext}{"50}
\undef\sum
\DeclareMathOperator*{\sum}{\mathchoice{\sum@d}{\sum@t}{\sum@t}{\sum@t}}
\DeclareMathOperator*{\bigsum}{\mathlarger{\mathlarger{\sum@d}}}

% phi computer modern
\DeclareMathAlphabet{\gk@mf}{OT1}{cmr}{m}{n}
\let\old@Phi\Phi
\def\Phi{\gk@mf{\old@Phi}}

% união elem por elem
\DeclareMathOperator{\wcup}{\mathaccent\cdot\cup}

% familia de conjuntos
\undef\fam
\DeclareMathAlphabet{\fam}{OMS}{cmsy}{m}{n}

% alguns símbolos
\undef\natural
\DeclareMathOperator{\real}{\mathbb{R}}
\DeclareMathOperator{\natural}{\mathbb{N}}
\DeclareMathOperator{\integer}{\mathbb{Z}}
\DeclareMathOperator{\complex}{\mathbb{C}}
\DeclareMathOperator{\rational}{\mathbb{Q}}
\def\symdif{\mathrel{\triangle}}
\def\midd{\;\middle|\;}
% \def\pow{\mathcal{P}}

% operações com mais espaçamento
\def\cupp{\mathbin{\,\cup\,}}
\def\capp{\mathbin{\,\cap\,}}

% alguns operadores
\DeclareMathOperator{\Dom}{Dom}
\DeclareMathOperator{\Img}{Im}
\DeclareMathOperator{\modulo}{~mod~}

% marcadores de operadores
\def\inv{^{-1}}
\def\rel#1{\use@invr{\!\mathrel{#1}\!}}
\def\nrel#1{\use@invr{\!\centernot{#1}\!}}
\def\dmod#1{\ (\mathrm{mod}\ #1)}
\def\cgc#1{{\textnormal{[}#1\textnormal{]}}}
\def\cgp#1{{\textnormal{(}#1\textnormal{)}}}

% marcador com inverso reduzido
\def\use@invr#1{%
    \begingroup%
        \edef\inv{\inv\!}%
        #1%
    \endgroup%
}

% delimiters
\def\abs#1{{\lvert\,#1\,\rvert}}
% \DeclarePairedDelimiter{\abs}{\lvert}{\:\rvert}

\makeatother


% math display skip
\newcommand{\reducemathskip}[1][0.5em]{%
    \setlength{\abovedisplayskip}{1pt}%
    \setlength{\belowdisplayskip}{#1}%
    \setlength{\abovedisplayshortskip}{#1}%
    \setlength{\belowdisplayshortskip}{#1}%
}

% url linking problems
\def\url#1{\href{#1}{\texttt{#1}}}
% vermelho
\def\red#1{\textcolor{red}{#1}}

\def\lmref#1{\thmref[lema ]{#1}}

\usepackage{xparse, caption, booktabs}
\usepackage[hidelinks]{hyperref}
\usepackage[nameinlink, brazilian]{cleveref}
\crefformat{equation}{#2eq.~#1#3}
\crefformat{definition}{#2def.~#1#3}
\crefformat{proof}{#2dem.~#1#3}
\usepackage[section, newfloat]{minted}
\definecolor{sepia}{RGB}{252,246,226}
\setminted{
    bgcolor = sepia,
    % style   = pastie,
    frame   = leftline,
    autogobble,
    samepage,
    python3,
}
\setmintedinline{
    bgcolor={}
}

\theoremstyle{plain}
\newtheorem*{hypothesis}{Hipótese}
\newtheorem*{theorem}{Teorema}
\newtheorem*{hypothesisf}{Hipótese Fortalecida}

\newtheoremstyle{definicao}% name of the style to be used
  {}% measure of space to leave above the theorem. E.g.: 3pt
  {}% measure of space to leave below the theorem. E.g.: 3pt
  {}% name of font to use in the body of the theorem
  {}% measure of space to indent
  {\bf}% name of head font
  {:}% punctuation between head and body
  {.8em}% space after theorem head; " " = normal interword space
  {\thmnote{\textbf{#3}}}% Manually specify head
\theoremstyle{definicao}
\newtheorem*{definition}{Definição}

\NewDocumentCommand{\seq}{ s m O{n} O{\in\natural} }
    {\IfBooleanTF{#1}
        {\ensuremath{\left({#2}_{#3}\right)}}
        {\ensuremath{\left({#2}_{#3}\right)_{{#3}{#4}}}}}


\usepackage{titling, titlesec, enumitem}
% \usepackage{algorithmic}
\usepackage{clrscode3e, xspace}
\title{\vspace{-2.5cm}\Large Lista de Exercícios Avaliativa 6 \\ \normalsize MC458 - 2s2020 - Tiago de Paula Alves - 187679}
\preauthor{}\author{}\postauthor{}
\predate{}\date{}\postdate{}
\posttitle{\par\end{center}\vskip-1em}

\titleformat{\section}{\large\bfseries}{\thesection}{.8em}{}
\titlespacing*{\section}{0pt}{.5em plus .2em minus .2em}{.5em plus .2em}

\newlist{casos}{enumerate}{2}
\setlist[casos]{wide,labelwidth={\parindent},listparindent={\parindent},parsep={\parskip},topsep={0pt},label=\textbf{Caso \arabic*}:}
\setlist[casos,2]{label=\textbf{Caso \arabic{casosi}\alph*}:}

\newlist{ncasos}{description}{2}
\setlist[ncasos]{wide,listparindent={\parindent},parsep={\parskip},topsep={0pt}}

\titleformat{\section}[runin]
    {\titlerule{}\vspace{1ex}\normalfont\Large\bfseries}{}{1em}{}[.]
\titleformat{\subsection}[runin]
    {\normalfont\large\bfseries}{}{1em}{}[)]

% linha final da página ou seção
\newcommand{\docline}[1][\\]{%
    #1\noindent\rule{\textwidth}{0.4pt}%
    \pagebreak%
}
\newcommand{\itemdsep}{
    \noindent\hfil\rule{0.5\textwidth}{.2pt}\hfil
    \vskip1em
}


\usepackage{tikz}
\usetikzlibrary{calc,trees,positioning,arrows,fit,shapes,calc}

\DeclareMathSymbol{\mlq}{\mathord}{operators}{``}
\DeclareMathSymbol{\mrq}{\mathord}{operators}{`'}
\def\gets{~\leftarrow~}

\usepackage{fancyhdr}
\pagestyle{empty}

% \usepackage{showframe}
\begin{document}

    \maketitle
    \thispagestyle{empty}

    % \noindent\rule{\textwidth}{0.4pt}
    % \begin{center}\Large\vskip-0.5em
    %     CORRIGIR A QUESTÃO \textbf{???}.
    % \end{center}

    \section{1}
    \begingroup
        \begin{theorem}[subestrutura ótima]
    Seja $C = [c_1, \ldots, c_n]$ uma lista de $n$ cortes com número de trilhos mínimo e seja $C' = [c_1, \ldots, c_m]$, com $m \leq n$, a sublista contínua com soma mais próxima de $M$. Então, a sublista $C - C' = [c_{m + 1}, \ldots, c_n]$ são cortes com número mínimo de trilhos.
\end{theorem}

\begin{theorem}[escolha gulosa]
    ...
\end{theorem}

    \endgroup

    \docline[]

    \section{2}
    \begingroup
        Considere um deslocamento $T = (t_1, \ldots, t_n)$ como uma sequência, com $t_i = -1$ ou $t_i = +1$ para todo $i \leq n$, de forma que $P_T = (p_1 + t_1 k, \ldots, p_n + t_n k)$ é uma das $2^n$ possíveis sequências construídas a partir de $P$ e $k$.

\itemdsep

\begin{theorem}[subestrutura ótima]
    Seja $T = (t_1, t_2, \ldots, t_n)$ um deslocamento que gera a sequência $P_T$ com desequilíbrio mínimo. Se a subsequência $P' = (p_2, \ldots, p_n)$ contém o máximo e o mínimo de $P$, então o deslocamento $T' = (t_2, \ldots, t_n)$ também gera uma sequência $P'_{T'}$ de desequilíbrio mínimo.
\end{theorem}

\itemdsep

\begin{theorem}[escolha gulosa]
    Seja $T = (t_1, \ldots, t_n)$ um deslocamento que gera a sequência $P_T$ com desequilíbrio mínimo, $p_{\max} = \max_{1 \leq i \leq n} p_i$ e $p_{\min} = \min_{1 \leq i \leq n} p_i$. Então,
    \[
        t_1 = \begin{cases}
            +1, & \text{se } \abs{p_1 - p_{\max}} < \abs{p_1 - p_{\min}} \\
            -1, & \text{caso contrário}
        \end{cases}
    \]
\end{theorem}

\itemdsep

\begin{codebox}
    \Procname{$\proc{Desequilíbrio-Mínimo}(P, n)$}
    \li \kw{se} $n \isequal 1$ \kw{então}
        \Do
    \li     \kw{retorna} 0
        \End
    \li
    \li $\id{p-min} \gets P[1]$
    \li $\id{p-max} \gets P[1]$
    \li \kw{para} $i \gets 2$ \kw{até} $n$
        \Do
    \li     $\id{p-min} \gets \min(\id{p-min}, P[i])$
    \li     $\id{p-max} \gets \max(\id{p-max}, P[i])$
        \End
    \li
    \li $\id{p-inf} \gets \id{p-min}$
    \li $\id{p-sup} \gets \id{p-max}$
    \li \kw{para} $i \gets 1$ \kw{até} $n$
        \Do
    \li     \kw{se} $\id{p-max} - P[i] > P[i] - \id{p-min}$ \kw{então}
            \Do
    \li         $\id{p-inf} \gets \min(\id{p-inf}, P[i])$
            \End
    \li     \kw{senão}
            \Do
    \li         $\id{p-sup} \gets \min(\id{p-sup}, P[i])$
            \End
        \End
    \li
    \li $\id{max-diff} \gets \min(\id{p-sup} - \id{p-min}, \id{p-max} - \id{p-inf})$
    \li \kw{retorna} $\lceil \id{max-diff} / 2 \rceil$
\end{codebox}

\itemdsep

Podemos ver que todos os laços na função executam no máximo $n$ vezes. Então, a complexidade de tempo do algoritmo é $T(n) \in \Theta(n)$. Como não é usado nenhum armazenamento adicional, o espaço tem complexidade constante, $E(n) \in \Theta(1)$.

    \endgroup

    \docline[]

    \section{3}
    \begingroup
        \begin{theorem}[subestrutura ótima]
    Sejam $k > 0$ e $n$ um inteiro com representação mínima $T = [t_0, t_1, \ldots, t_k]$. Então, $[t_1, \ldots, t_k]$ é uma representação mínima de $\dfrac{n - t_0}{3}$.
\end{theorem}

\begin{proof}[\textbf{Demonstração}]
    Considere $m = (n - t_0) / 3$ e sua representação mínima $R = [r_0, r_1, \ldots, r_{l^*}]$ de tamanho $l^*$. Suponha agora uma representação $S = [s_0, \ldots, s_l]$ de $m$ que não é ótima, isto é, $l > l^*$. Assim, temos que:
    \begin{align*}
        n &= 3 m + t_0 \\
        &= t_0 + 3 \left(r_0 + r_1 \cdot 3 + \cdots + r_{l^*} \cdot 3^l\right) \\
        &= t_0 + r_0 \cdot 3 + r_1 \cdot 3^2 + \cdots + r_{l^*} \cdot 3^{l^*+1}
    \end{align*}
    Então, temos a representação $R' = [t_0, r_0, \ldots, r_{l^*}]$ para $n$. Da mesma forma, temos que $S' = [t_0, s_0, \ldots, s_l]$ representando o mesmo número. Entretanto,
    \[
        \abs{R'} = \abs{R} + 1 = l^* + 2 < l + 2 = \abs{S} + 1 = \abs{S'}
    \]

    Ou seja, a representação $S'$ não é ótima. Portanto, pela contrapositiva, para qualquer representação $A$ de $n$, se ela for mínima, $A - [t_0]$ será uma representação mínima de $m$. Como $T$ é ótima, o teorema segue.
\end{proof}

% ~

% Considere que $C_i$ é o menor custo de um percurso da aldeia $A_i$ para a $A_n$, sendo $1 \leq i < n$. Como os subpercursos são ótimos, podemos considerar todas as possíveis paradas $A_{i < j \leq n}$ com seus percursos ótimos até $A_n$, tomando o menor deles como parte do percurso ótimo de $A_i$. Note que se $i = n$, então não resta nenhuma aldeia no percurso, ou seja, $C_i = 0$. Assim,
% \begin{align*}
%     C_i &= \min_{i < j \leq n}\left\{t_{i, j} + C_j\right\} \\
%     C_n &= 0
% \end{align*}

% Nessa relação, o custo ótimo $C_i$ depende apenas dos custos $C_j$ das aldeias seguintes, com $i < j$, já que $t_{i, j} > 0$ e não é possível alugar uma canoa de $A_{j > i}$ para $A_i$. Assim, podemos calcular os custo a partir da última aldeia $A_n$, sem necessidade de recursão ou memorização.

% \itemdsep
% \subsection{b}

\begin{codebox}
    \Procname{$\proc{Mínimo-Dígitos}(n)$}
    \li $k \gets -1$
    \li
    \li \kw{enquanto} $n > 0$
        \Do
    \li     $k \gets k + 1$
    \li     \kw{se} $k \modulo 3 \isequal 0$ \kw{então}
            \Do
    \li         $n \gets n / 3$
            \End
    \li     \kw{senão}, \kw{se} $k \modulo 3 \isequal 1$ \kw{então}
            \Do
    \li         $n \gets (n - 1) / 3$
            \End
    \li     \kw{senão} ~ ~ ~ \Comment{$k \modulo 3 \isequal 2$}
            \Do
    \li         $n \gets (n + 1) / 3$
            \End
        \End
    \li
    \li \kw{retorna} $k$
\end{codebox}

\begin{codebox}
    \Procname{$\proc{Mínimo-Dígitos}(n)$}
    \li \kw{se} $n \isequal 1$ \kw{então}
        \Do
    \li     \kw{retorna} $0$
        \End
    \li
    \li \kw{senão}, \kw{se} $k \modulo 3 \isequal 0$ \kw{então}
        \Do
    \li     \kw{retorna} $\proc{Mínimo-Dígitos}(n / 3)$
        \End
    \li
    \li \kw{senão}, \kw{se} $k \modulo 3 \isequal 1$ \kw{então}
        \Do
    \li     \kw{retorna} $\proc{Mínimo-Dígitos}((n - 1) / 3)$
        \End
    \li
    \li \kw{senão} ~ ~ ~ \Comment{$k \modulo 3 \isequal 2$}
        \Do
    \li     \kw{retorna} $\proc{Mínimo-Dígitos}((n + 1) / 3)$
        \End
\end{codebox}

% \itemdsep
% \subsection{c}

% Vamos considerar que as linhas \ref{linha:a1:1} e \ref{linha:a1:2} executam em tempo constante $a_1$, que a \ref{linha:a2:1} e a \ref{linha:a2:2} executam em $a_2$ e que \ref{linha:a3:1} e \ref{linha:a3:2} são em tempo $a_3$. Assim, podemos descrever o tempo de execução do algoritmo por:
% \begin{align*}
%     T(n) &= a_1 + \sum_{i = 1}^{n - 1}\left(a_2 + \sum_{j = i + 1}^{n - 1} a_3\right) \\
%     &= a_1 + \sum_{i = 1}^{n - 1} a_2 + a_3 \sum_{i = 1}^{n - 1} n - a_3 \sum_{i = 1}^{n - 1} i - a_3 \sum_{i = 1}^{n - 1} 1 \\
%     &= a_1 + a_2 (n - 1) + a_3 n (n - 1) - a_3 \frac{n (n - 1)}{2} - a_3 (n - 1) \\
%     &= \frac{a_3}{2} n^2 + \frac{2 a_2 - 3 a_3}{2} n + a_1 - a_2 + a_3
% \end{align*}

% Ou seja, $T(n) \in \Theta\left(n^2\right)$, como requerido. Além disso, o único espaço adicional é do vetor $C$ de tamanho $n - 1$. Então, a complexidade de espaço será dada por:
% \begin{align*}
%     E(n) &= n - 1 + \Theta(1) = \Theta(n)
% \end{align*}

\[
    T(n) = T(n / 3) + \Theta(1) \in \Theta(\log(n))
\]

    \endgroup

    \docline[]

\end{document}
